
Let an SIS commit to a {0,..,v} vector mean a commit to a {0,..,v}-valued vector.
Start with the prover claiming an opening to an SIS commit to a vector of Euclidean norm.
Reduce instead to prover claiming openings to s SIS commits to {0,..,w-1} vectors (w chosen later).
Eg, one (inefficient) way is using binary decomposition into {0,1} vectors of the same length.
Then use the following protocol for proof composition.

1.
Start with prover claiming openings to s SIS commits to {0,..,w-1} vectors.
Use, say, binary decomposition to somehow split each commit into w commits to {0,1} vectors.
The new vectors will be a 2^w/w^s factor shorter in length.
2.
Use a succinct proof technique, such as sumcheck, to reduce certain checks on the w vectors to evaluating them as polynomials.
3.
A. Check that each vector contains only {0,1} values.
B. Reconstruct the original vectors from the binary decomposition, and check that the original vectors indeed open to the original commits.
The two checks together also ensure the original vectors have values in {0,..,w-1}.
4.
Use a random w by s binary matrix (with no column all 1's) as a challenge to both the openings of the new vectors and also their polynomial evaluations.
5.
Multiplying the w commits to {0,1} vectors by this matrix the verifier gets s commits to {0,..,w-1} vectors (to permit all 1's columns permits {0,..,w} vectors which complicates analysis and binary decomposition).
6.
Prover sends the resulting s vectors to the verifier who checks they only contain {0,..,w-1} values and satisfy polynomial evaluations.

To extract knowledge for a single run of the protocol, use the heavy table argument, which requires 2 valid challenge/response pairs to extract a (binding) weak opening for each of the w {0,1} vectors.
The extraction doesn't guarantee they are {0,1} vectors, but checking this was reduced to checking polynomial evaluation, so with all but negl(s) soundness error, vector values are all {0,1}.
Finally use these w {0,1} vectors to reconstruct (wrt binary decomposition) the original s {0,..,w-1} vectors of 2^w/w^s times the length.

When composing the proof there are log_{2^w/w^s}(n) rounds with n the original vector length.
Extract using the forking lemma (like in bulletproofs), where the tree has radix 2w and depth no more than log_{2w}(n) to keep the tree size polynomial in s.
To achieve this depth we need 2^w/w^s >= 2w, that is w >= (s+1)log(w)+1 (log base 2), for which w = 1.5s*log(s) suffices when s > 123.
So each of the log_{2^w/w^s}(n) rounds the prover sends 1.5s*log(s) commits, along with whatever data is sent in the inner protocol (eg sumcheck) which has its own rounds.

There's a more efficient (and complex) way using RSIS with a cyclotomic polynomial that fully splits, and using Galois-rotated challenges from the paper `practical product proofs.'
Doing so means inverting perspective of the the commitment scheme, where rather than supporting commits to small norm polynomials, it supports commits to vectors in Z_p which interpolate to small norm polynomials.
Sampling challenges as done in the paper (maybe room for improvement) leads to log_{ws/4}(n) rounds again sending w commits each round where w = 6*log(s)+26 for s > 35.
