
We introduce cyclotomic extensions and examine the special cases of extending $\Q$ and $\F_q$, the two most common base fields for such extensions.
We also introduce cyclotomic polynomials and explore some basic properties relevant to cyclotomic extensions.
Two general sources used but not cited elsewhere are \cite{Mor96} and \cite{Conrad}.

\begin{itemize}
    \item
    Let $\phi$ denote Euler's totient function, that is $\phi(n)$ counts the numbers between $1$ and $n$ that are coprime with $n$.
    \item
    Let $\ord_n(x)$ denote the order of element $x\in(\Z/n\Z)^\times$ in the multiplicative group $(\Z/n\Z)^\times$.
    \item
    Let the notation $(a,b)$ for $a,b\in\Z$ denote the greatest common divisor of $a$ and $b$, that is $\text{gcd}(a,b)$.
\end{itemize}

For positive integer $n$, the $n$'th complex roots of unity are the roots of the polynomial $X^n-1$ over $\C$, namely $\zeta_k = e^{2\pi i(k/n)}$ for $1\leq k\leq n$ which form a cyclic group of order $n$.
The \emph{primitive} $n$'th complex roots of unity are those $\zeta_k$ for which $(k,n)=1$, of which there are $\phi(n)$ by definition of $\phi$.

We may extend the notion of complex roots of unity to roots of unity over any field in which $X^n-1$ has $n$ distinct roots.
These roots are the  $n$'th roots of unity denoted $\mu_n=\{\zeta_k\}_{k\in[n]}$.
They form a cyclic subgroup of the field's multiplicative group with generators any roots of $X^n-1$ not roots of $X^m-1$ for any $m<n$.
Such an element of $\mu_n$ must exist, otherwise some $X^m-1$ with $m|n$ would have $n>m$ roots.
If $\zeta\in\mu_n$ is a generator, then $\zeta^k\in\mu_n$ has order $n/(n,k)$ and is therefore also a generator precisely when $k$ and $n$ are coprime.
There are $\phi(n)$ generators of $\mu_n$ by definition of $\phi$.
We call these generators the \emph{primitive} $n$'th roots of unity.


\section{Extensions in general}

After presenting the definition of cyclotomic extensions we immediately show such extensions are Galois.
The Galois groups of $n$'th cyclotomic extensions are then shown isomorphic to subgroups of $(\Z/n\Z)^\times$
At this point we introduce cyclotomic polynomials which we use in the following two sections, but before proceeding to those sections we prove two central properties of cyclotomic polynomials.
The first is the fact that all cyclotomic polynomials have integer coefficients, and as such can be interpreted over any field.
The second shows that all $n$'th cyclotomic polynomials are equivalent to the extent of such interpretation.

Cyclotomic extensions are had by adjoining a complete set of $n$'th roots of unity to a field.
A central assumption we carry throughout exploration of cyclotomic extensions is that if the characteristic of the field is non-zero (i.e. prime), then it does not divide $n$, otherwise the field cannot hold all $n$'th roots of unity.
To see this, suppose the field has characteristic $p$ and $n=m\cdot p^\ell$.
Since $p$ is prime we apply the Frobenius endomorphism to see $X^n-1=(X^m-1)^{p^\ell}$.
Then for any $\zeta\in\mu_n$, $\zeta^n=1$ implies $\zeta^m=1$ with $m<n$ and therefore no element of $\mu_n$ can function as a primitive $n$'th root of unity in the field.
So if we are to assume $\mu_n$ behaves as a complete set of $n$'th roots of unity in a field we must assume the field has characteristic zero or characteristic not dividing $n$.
Henceforth we implicitly make this assumption.

\begin{definition}
    A $n$'th \emph{cyclotomic} extension of a field $F$ is obtained by adjoining a set of $n$'th roots of unity $\mu_n$ to $F$.
    The roots of unity $\mu_n$ may belong to any extension field of $F$.
    The field $F(\mu_n)$ is called an $n$'th cyclotomic field.
\end{definition}
Note that adjoining all $n$'th roots $\mu_n$ is equivalent to adjoining any generator of the set $\mu_n$, that is any primitive $n$'th root of unity.

Consider the unique polynomial vanishing on exactly the $n$ distinct elements $\mu_n$, that is the polynomial $X^n-1$.
Since $F(\mu_n)$ contains all $n$ roots of $X^n-1$ (that is $\mu_n$), it's clear that $X^n-1$ splits in $F(\mu_n)$.
Moreover, since the $n$ elements of $\mu_n$ are distinct, $X^n-1$ splits into distinct linear factors in $F(\mu_n)$.
With such splitting in mind we now show the extension $F(\mu_n)/F$ is Galois.

\begin{theorem}[F(mu_n)/F is Galois]{$F(\mu_n)/F$ is Galois}
    For a field $F$, the extension $F(\mu_n)/F$ is Galois.

    \proof
    We show normality and separability of this extension, that is the criteria for an extension to be Galois.
    In both parts we use the following lemma, the proof of which is not relevant for our purposes.
    Variations of the lemma can be found in many introductory texts on field theory (e.g. \cite{Mitchell}).
    \begin{lemma}
        Suppose $E/F$ is an algebraic field extension.
        If the minimal polynomial over $F$ of \emph{some} $\alpha_0\in E$ splits in $E$, then the minimal polynomial over $F$ of \emph{every} $\alpha_i\in E$ splits in $E$.
        Moreover, if the minimal polynomial of $\alpha_0$ is separable, then the minimal polynomial of \emph{every} $\alpha_i$ is separable.
    \end{lemma}

    Fix some $\zeta\in\mu_n$.
    The minimal polynomial over $F$ of $\zeta$ is a factor of $X^n-1$.
    To see this, suppose dividing $X^n-1$ by the minimal polynomial $f_\zeta$ of $\zeta$ leaves non-zero remainder $r$.
    Then $r$ must be of degree less than $f_\zeta$ and vanish on $\zeta$ (since both $X^n-1$ and $f_\zeta$ vanish on $\zeta$).
    As such, $r$ contradicts the minimality of $f_\zeta$.

    \bold{Normality:}
    Since $X^n-1$ splits in $F(\mu_n)$, so does its factors, including the minimal polynomial of $\zeta\in F(\mu_n)$.
    Applying the lemma above we conclude the minimal polynomial over $F$ of \emph{every} element in $F(\mu_n)$ splits in $F(\mu_n)$, and thus the extension $F(\mu_n)/F$ is normal.

    \bold{Separability:}
    Since $X^n-1$ splits into \emph{distinct} linear factors in $F(\mu_n)$, so does its factors, including the minimal polynomial over $F$ of $\zeta\in F(\mu_n)$.
    Therefore the minimal polynomial of $\zeta$ is separable.
    Applying the lemma above we conclude the minimal polynomial over $F$ of \emph{every} element in $F(\mu_n)$ is separable, and thus the extension $F(\mu_n)/F$ is separable.
\end{theorem}

With $F(\mu_n)/F$ Galois, what can we say about its Galois group?
Recall from \link[Algebra]{&&algebra} that the Galois group of a Galois extension $F(\mu_n)/F$ is the group of automorphisms on $F(\mu_n)$ that fix $F$.
Since the field $F(\mu_n)$ is composed of $F$ and $\mu_n$, an automorphism on $F(\mu_n)$ that fixes $F$ is defined solely by its behavior on $\mu_n$.
For every $\sigma\in\Gal(F(\mu_n)/F$, the next theorem shows this behavior to be $\sigma(\zeta) = \zeta^k$ for all $\zeta\in\mu_n$ and some $k$ coprime with $n$.
Thus $\sigma$ may be defined piecewise on $F(\mu_n)$ as
\begin{equation}
    \sigma(x) =
    \begin{cases}
        x^k, & x \in\mu_n \\
        x, & x \in F
    \end{cases}
\end{equation}

\begin{theorem}[Gal(F(mu_n)/F) is isomorphic to a subgroup of (Z/nZ)^*]{$\Gal(F(\mu_n)/F)$ is isomorphic to a subgroup of $(\Z/n\Z)^\times$}
    Each automorphism of $\Gal(F(\mu_n)/F)$ is determined by some integer coprime with $n$.
    Let $\sigma_k$ denote the automorphism determined by integer $k$ with $(k,n)=1$.
    Then $\sigma_k(\zeta)=\zeta^k$ for all $\zeta\in\mu_n$.

    The map $\sigma_k\to(k+\Z)$ of signature $\Gal(F(\mu_n)/F)\to(\Z/n\Z)^\times$ is an injective group homomorphism.
    Therefore, $\Gal(F(\mu_n)/F)$ is isomorphic to a subgroup of $(\Z/n\Z)^\times$.

    \proof
    Fix some primitive root of unity $\zeta_0\in\mu_n$, and recall that all generators of $\mu_n$ have the form $\zeta_0^k$ for some $k$ coprime with $n$.

    Since the automorphisms $\Gal(F(\mu_n)/F)$ of the field $F(\mu_n)$ fix the subfield $F$, they necessarily map $F(\mu_n)\setminus F = \mu_n$ to itself.
    Moreover, since they are automorphisms of the multiplicative group $F(\mu_n)^\times$, they are group automorphisms of $\mu_n$.
    As any automorphism of the group $\mu_n$ must map generators to generators, for $\sigma\in\Gal(F(\mu_n)/F)$ we must have $\sigma(\zeta_0)=\zeta_0^k$ for some $k$ coprime with $n$.
    With any element of $\mu_n$ taking the form $\zeta_0^t$ for some $t$, we have
    \begin{equation}
        \sigma(\zeta_0^t) = \sigma(\zeta_0)^t = (\zeta_0^k)^t = (\zeta_0^t)^k
    \end{equation}
    Thus $\sigma$ is defined by raising all elements of $\mu_n$ to $k$.
    We associate $\sigma$ with $k$ and write $\sigma_k$.
    To see the subscript notation $\sigma_k$ is unique modulo $n$, suppose $\sigma_{k'}=\sigma_k$.
    Then $\zeta_0^{k-k'}=1$ and since $\zeta_0$ has order $n$, it must be that $k-k'$ divides $n$.

    We now show the map $\sigma_k\to(k+\Z)$ is an injective group homomorphism from $\Gal(F(\mu_n)/F)$ to $(\Z/n\Z)^\times$.
    The fact that $(k,n)=1$ for all $\sigma_k$ implies the image of $\sigma_k$ is contained in $(\Z/n\Z)^\times$.
    For $\sigma_k,\sigma_{k'}\in\Gal(F(\mu_n)/F)$ and $\zeta\in\mu_n$ we have 
    \begin{equation}
        \sigma_k\circ\sigma_{k'}(\zeta) = \sigma_k(\sigma_{k'}(\zeta))
        = \zeta^{kk'} = \sigma_{kk'}
    \end{equation}
    Denoting the map by $\psi$ we readily see it is a homomorphism as
    \begin{equation}
        \psi(\sigma_k\circ\sigma_{k'}) = \psi(\sigma_{kk'}) = kk' = \psi(\sigma_k)\cdot\psi(\sigma_{k'})
    \end{equation}
    The kernel of the homomorphism is all $\sigma_k$ such that $k\equiv1(\bmod n)$ in which case $\sigma_k$ is the identity function, that is the identity element of the Galois group.
    The homomorphism is therefore injective.
\end{theorem}

We now introduce cyclotomic polynomials for their roles in cyclotomic extensions of $\Q$ and $\F_q$ as explored in the next two sections.
The $n$'th cyclotomic polynomial is defined as the monic, non-zero polynomial vanishing on the primitive $n$'th roots of unity in $\mu_n$.
Suppose $\zeta_0\in\mu_n$ is a primitive root of unity.
\begin{equation}
    \Phi(X)\coloneqq\prod_{\substack{
        \text{$n$'th primitive}\\
        \text{roots of unity $\zeta$}
    }} (X - \zeta)
    = \prod_{\substack{1\leq k\leq n\\\text{gcd}(k,n)=1}} (X-\zeta_0^k)
\end{equation}
Since there are $\phi(n)$ primitive $n$'th roots of unity, $\Phi_n$ has degree $\phi(n)$.

We may partition the $n$'th roots of unity into all sets of primitive $d$'th roots of unity, one set for every $d$ dividing $n$.
To see this, note that every $n$'th root of unity generates a subgroup of order $d$ and is therefore a primitive $d$'th root of unity for exactly one $d$ dividing $n$.
On the other hand, since $d$ divides $n$ every primitive $d$'th root of unity is also an $n$'th root of unity.
With this partition in mind we may represent $X^n-1$ as the product of all cyclotomic polynomials $\Phi_d(X)$ for $d$ a divisor of $n$.
\begin{equation}
    X^n - 1 = \prod_{d|n} \Phi_d(X)
\end{equation}
We may regard this equation as an implicit, recursive definition of cyclotomic polynomials, written explicitly as
\begin{equation}
    \Phi_n(X) = \big(X^n - 1\big) \Bigg/ \prod_{\substack{d|n\\d\neq n}} \Phi_d(X)
\end{equation}

A potentially surprising property of cyclotomic polynomials is that they have integer coefficients.
Indeed, regardless the $n$'th roots of unity $\mu_n$ used to define $\Phi_n$, even if the \emph{irrational} complex roots of unity, expanding the polynomial $\Phi_n$ into coefficient form always yields integer coefficients.
Note that integer coefficients can alternatively be interpreted as belonging to any field by identifying them with multiples of the field's additive identity.
Thus when considering $\Phi_n$ over $F(\mu_n)$ we may say $\Phi_n$ has coefficients over $F$.

\begin{theorem}[Phi_n(X) is in Z[X]]{$\Phi_n[X]\in\Z[X]$}
    The $n$'th cyclotomic polynomial $\Phi_n$ has integer coefficients.

    \proof
    We show this by induction using the recursive definition of cyclotomic polynomials.
    The base case in clear with $\Phi_1(X)=X-1\in\Z[X]$.
    For induction consider the equation
    \begin{equation}
        X^n-1 = \Phi_n(X) \cdot
        \prod_{\substack{d|n\\d\neq n}}\Phi_d(X) = \Phi_n(X)\cdot\gamma(X)
    \end{equation}
    The polynomial $X^n-1$ has integer coefficients, and by induction so does $\gamma$.
    Note that $\gamma$ is also monic as it is the product of monic polynomials.
    Then one may observe that in dividing $X^n-1$ by $\gamma$, the long division algorithm in $\Z[X]$ must yield a polynomial also in $\Z[X]$.
    For a more explicit argument, suppose the long division algorithm in $\Z[X]$ yields quotient $q\in\Z[X]$ and remainder $r\in\Z[X]$ such that $X^n-1=q(X)\gamma(X)+r(X)$.
    Then $q(X)\gamma(X)+r(X)=\Phi_n(X)\gamma(X)$ forces $r(x)=\gamma(X)(\Phi_n(X)-q(X))$ to be zero, otherwise $\deg(r)\geq\deg(\gamma)$.
    Therefore, $\Phi_n=q\in\Z[X]$.
\end{theorem}

We can strengthen the previous result further.
We can in fact say that all cyclotomic polynomials are equivalent in the sense that two $n$'th cyclotomic polynomials defined over different fields share the same roots when interpreted over the same field.
As mentioned previously, such cross-field interpretation is possible because cyclotomic polynomials have integer coefficients.
As a result, a lot can be said about cyclotomic polynomials (almost) oblivious to the field over which they're defined.
We examine some such field-independent properties in \link[Cyclotomic polynomials]{&cyclotomic polynomials}.

We demonstrate equivalence between $n$'th cyclotomic polynomials by showing they are all equivalent to the $n$'th cyclotomic polynomial over $\C$.
Consider the $n$'th cyclotomic polynomial in $\C[X]$ defined as vanishing exactly on $e^{2\pi i(k/n)}$ for $1\leq k\leq n$ with $(n,k)=1$.
Interpreting this polynomial over any $n$'th cyclotomic field $F(\mu_n)$, we argue it vanishes on the primitive $n$'th roots of unity in $\mu_n$.
Therefore, the following two polynomials are equal:
\begin{itemize}
    \item
    The $n$'th cyclotomic polynomial defined over $F(\mu_n)$ as vanishing on exactly the primitive $n$'th roots of unity in $\mu_n$.
    \item
    The $n$'th cyclotomic polynomial defined over $\C$ and interpreted over $F(\mu_n)$.
\end{itemize}
By the transitivity of equivalence, we may then say any two cyclotomic polynomials defined over different fields are equivalent `up to interpretation.'

\begin{theorem}[Phi_n in C[X] equates to Phi_n in F(mu_n)[X]]{$\Phi_n\in\C[X]\equiv\Phi_n\in F(\mu_n)[X]$}
    Let $\Phi_n$ be the $n$'th cyclotomic polynomial in $\C[X]$.
    Let $F(\mu_n)$ be an $n$'th cyclotomic field.
    Then $\overline{\Phi_n}\in F(\mu_n)[X]$ vanishes on precisely the primitive $n$'th roots of unity in $\mu_n$, where $\overline{\Phi_n}$ is $\Phi_n$ interpreted over $\F(\mu_n)$.

    \proof
    Let $\overline{x}$ for $x\in\Z[X]$ denote $x\in F[X]$, that is $x$ from $\Z[X]$ interpreted in $F[X]$.
    Suppose division of $X^n-1$ by $\Phi_n$ in $\Z[X]$ yields quotient $q$ and remainder $r$.
    Then we may write $\overline{X^n-1} = \overline{\Phi_n}(X)\cdot\overline{q}(X)+\overline{r}(X)$ with relations $\deg(\overline{r})<\deg(\overline{\Phi_n})=\phi(n)$ still holding.
    That is, as $\Phi_n$ divides $X^n-1$ in $\Z[X]$, so $\overline{\Phi_n}$ divides $\overline{X^n-1}=X^n-\overline{1}$ in $F[X]$.
    Note that $X^n-\overline{1}$ splits in $F(\mu_n)$ with roots $\mu_n$, and therefore its factor $\overline{\Phi_n}$ also splits in $F(\mu_n)$ with roots some subset of $\mu_n$.
    Since the subset must be of size $\phi(n)=\deg(\overline{\Phi_n})$, we are left to show that all roots of $\overline{\Phi_n}$ are primitive $n$'th roots of unity.

    Supposing $\overline{\Phi_n}(\zeta)=0$ for $\zeta\in\mu_n$, we must show $\zeta$ is a primitive $n$'th root of unity.
    We will refer to the following equation as the `cyclotomic decomposition' of $\zeta^m-\overline{1}$ for some $m$.
    \begin{equation}
        \zeta^m-\overline{1} = \prod_{d|m} \overline{\Phi_d}(\zeta)
    \end{equation}
    Suppose $\zeta$ is not a primitive $n$'th root of unity.
    Then $\zeta$ is a primitive $m$'th root of unity for some proper divisor $m$ of $n$.
    The cyclotomic decomposition of $\zeta^m-\overline{1}$ suggests $\overline{\Phi_d}(\zeta)=0$ for some $d|m$ which is a proper divisor of $n$.
    We now contradict this statement by recalling that $X^n-\overline{1}$ has no repeated roots, and therefore $\zeta$ can only vanish on one of its factors.
    That is, $\overline{\Phi_d}(\zeta)=0$ can only occur for one $d|n$ in the cyclotomic decomposition of $\zeta^n-\overline{1}$.
    Since we assume $\overline{\Phi_n}(\zeta)=0$, it must be that $\overline{\Phi_d}(\zeta)\neq0$ for all proper divisors $d$ of $n$.
\end{theorem}


\section[Extensions of Q]{Extensions of $\Q$}

First we examine the reducibility of $\Phi_n$ over $\Q$, then see what it means for the Galois group $\Q(\mu_n)/\Q$.
We also mention the ring of integers of $\Q(\mu_n)$.
Note that since the field $\Q$ has characteristic zero, we have met the assumption stated at the start of the prior section to ensure $\mu_n$ functions as a full set of $n$'th roots of unity over $\Q$.

\begin{theorem}[Phi_n is irreducible over Q]{$\Phi_n$ is irreducible over $\Q$}
    The $n$'th cyclotomic polynomial $\Phi_n$ is irreducible over $\Q$.

    \proof
    See \cite{Weintraub} for several of the most historical proofs.

    \proof
    Here we give a proof adapted from \cite{Chen06}.

    Suppose $\Phi_n(X)=f(x)g(x)\in\Z[X]$ with $f$ irreducible over $\Q$.
    We will prove that for all primes $p$ not dividing $n$, if primitive $n$'th root of unity $\zeta$ is a root of $f$, then so is $\zeta^p$.
    Therefore, for all $\phi(n)$ values $k$ with $(k,n)=1$ we can say $\zeta^k$ is also a root of $f$, and thus $f$ has degree $\phi(n)$ so it must be $\Phi_n$ itself.
    To justify the latter conclusion, consider how any $k$ has a prime decomposition, and to say $(k,n)=1$ is to say no prime in the decomposition divides $n$.
    To show $f$ vanishes on $\zeta^k$, we raise $\zeta$ consecutively to each instance of each prime in the decomposition.
    Each time we do so, we are raising a primitive $n$'th root of unity on which $f$ vanishes to a prime not dividing $n$, so we may conclude the result is another primitive $n$'th root of unity on which $f$ vanishes.

    Now we prove that for prime $p$ not diving $n$, $\zeta^p$ is a root of $f$.
    Since $\zeta^p$ is an $n$'th primitive root of unity, we have $\Phi_n(\zeta^p)=0$ in which case either $f(\zeta^p)=0$ or $g(\zeta^p)=0$.
    Suppose the latter for sake of contradiction.
    Since $f$ is irreducible it must be the minimal polynomial of $\zeta$ over $\Q$ and so must divide $g(x^p)$.
    Now $f$ and $g$ have rational coefficients because they are products of the minimal polynomials of the primitive $n$'th roots of unity over $\Q$.
    Then by a variant of what's known as \link[Gauss' Lemma]{jupiter.math.nycu.edu.tw/~weng/courses/alg_2007/Algebra\%202006-2/GaussLemma.pdf}, by the fact that $f(x)g(x)=\Phi_n(x)$ it must be that $f$ and $g$ and thus $g(x^p)$ have integer coefficients.
    With quotient $q\in\Q[X]$, $f\in\Z[X]\subseteq\Q[X]$ and $g(x^p)\in\Z[X]$, we apply Gauss' Lemma again to the equation $f(x)q(x)=g(x^p)$ and conclude $q\in\Z[X]$.
    We may then pass the equation $f(x)q(x)=g(x^p)$ through the homomorphism $\overline{\cdot}\colon\Z[X]\to\Z_p[X]$ reducing modulo $p$.
    With $\overline{g}(x^p)=\overline{g}(x)^p$ by the Frobenius automorphism, we have $\overline{f}(\zeta)\overline{q}(\zeta)=\overline{g}(\zeta)^p=0$.
    Considering the equation $\overline{g}(\zeta)^p=0$ in $\C$ we conclude $\overline{g}(\zeta)=0$ since $\C$ has no zero divisors.

    Now $\overline{\Phi_n}$ divides $\overline{X^n-1}=X^n-\overline{1}$ and $\overline{\Phi_n}=\overline{f}(X)\overline{g}(X)$, so we may write
    \begin{equation}
        X^n-\overline{1} = \frac{X^n-\overline{1}}{\overline{\Phi_n}(X)}
            \cdot\overline{f}(X)\cdot\overline{g}(X)
    \end{equation}
    By $\overline{f}(\zeta)=\overline{g}(\zeta)=0$ we see $\zeta$ is a double root of $X^n-\overline{1}$.
    But $X^n-\overline{1}$ has non-zero derivative $nX^n$ since we assume $p$ doesn't divide $n$, and therefore $X^n-\overline{1}$ has no repeated roots.
    Thus a contradiction.
\end{theorem}

The irreducibility of $\Phi_n$ over $\Q$ allows us to determine the degree of the extension $\Q(\mu_n)/\Q$.
Recall that $\Phi_n$ over $\Q(\mu_n)$ has $\phi(n)$ roots, namely the primitive $n$'th roots of unity.
Since $\Phi_n$ is irreducible over $\Q$, it must be the minimal polynomial for these primitive $n$'th roots of unity.
Therefore the extension $\Q(\mu_n)/\Q$ has degree $\phi(n)$.
To see this, note that the field $\Q(\mu_n)$ is equivalent to the field $\Q[X]/\Phi_n(X)$ with equality had by identifying $X$ with any primitive $n$'th root of unity.
The field $\Q[X]/\Phi_n(X)$ extends $\Q$ with degree $\deg(\Phi_n)=\phi(n)$ and the claim follows.

The identity $[\Q(\mu_n):\Q]=\phi(n)$ allows us to now fully determine the Galois group.
\begin{corollary}[Gal(Q(mu_n)/Q) is isomorphic to (Z/nZ)^*]{$\Gal(\Q(\mu_n)/\Q)\cong(\Z/nZ)^\times$}
    The Galois group $\Gal(\Q(\mu_n)/\Q)$ is isomorphic to the multiplicative group $(\Z/nZ)^\times$.

    \proof
    We will not prove it here, but a basic fact of Galois theory is that the order of the Galois group equals the degree of the Galois extension.
    Having just established that $\Q(\mu_n)/\Q$ has degree $\phi(n)$, we conclude
    \begin{equation}
        \big|\Gal(\Q(\mu_n)/\Q)\big| = \big[\Q(\mu_n) : \Q\big] = \phi(n)
    \end{equation}
    We previously showed that \link[$\Gal(\Q(\mu_n)/\Q)$ is isomorphic to a subgroup of $(\Z/n\Z)^\times$]{#Gal(F(mu_n)/F) is isomorphic to a subgroup of (Z/nZ)^*}.
    Since the group $(\Z/n\Z)^\times$ has order $\phi(n)$ we conclude the subgroup must be $(\Z/n\Z)^\times$ itself.
\end{corollary}

Before moving on to the case of finite fields, we mention another important theorem of cyclotomic extensions of $\Q$.
The ring of integers of the field $\Q(\mu_n)$ is the ring $\Z[\mu_n]$.
The ideals of the ring $\Z[\mu_n]$ are the `ideal lattices' at the foundation of modern lattice cryptography.
Note that $\Z[\mu_n]$ is equivalent to $\Z[X]/\Phi_n(X)$.

\begin{theorem}[O_{Q(mu_n)} = Z[mu_n]]{$\mathcal{O}_{\Q(\mu_n)}=\Z[\mu_n]$}
    The ring of integers of $\Q(\mu_n)$ is $\Z[\mu_n]$.

    \proof
    The proof is involved, see \cite{Nguyen}.
\end{theorem}


\section[Extensions of F_q]{Extensions of $\F_q$}

In contrast to the previous section, we first examine the Galois group $\F_q(\mu_n)/\F$, then see what it means for the reducibility of $\Phi_n$ over $\F_q$.
Since finite fields have non-zero characteristic, recall that if $\mu_n$ is to function as a full set of $n$'th roots of unity in a finite field, the field characteristic cannot divide $n$.
We continue to implicitly make this assumption below.

We already know that \link[$\Gal(F_q(\mu_n)/F_q)$ is isomorphic to a subgroup of $(\Z/n\Z)^\times$]{#Gal(F(mu_n)/F) is isomorphic to a subgroup of (Z/nZ)^*}.
We now argue this subgroup in generated by $q$.

\begin{theorem}[Gal(F_q(mu_n)/F_q) is isomorphic to <q+Z>]{$\Gal(\F_q(\mu_n)/\F_q)$ is isomorphic to $\langle q+\Z\rangle$}
    For prime power $q$, the group $\Gal(\F_q(\mu_n)/\F_q)$ is isomorphic to the subgroup $\langle q+\Z\rangle$ of $(\Z/n\Z)^\times$ and thus has order $\ord_n(q)$.

    \proof
    We will show the $q$'th power map is an element of the Galois group, and then show the $q$'th power map in fact generates the Galois group.
    Using this knowledge we determine the image of the Galois group in $(\Z/n\Z)^\times$.

    Before proceeding we mention the Frobenius map.
    Suppose $q=p^\ell$.
    The Frobenius map is the $p$'th power map, which is an endomorphism of $\F_q(\mu_n)$.
    In fact, it is an automorphism of $\F_q(\mu_n)$ because its injective and $\F_q(\mu_n)$ is finite.
    To see injectivity, note the kernel is all $x^\in\F_q(\mu_n)$ such that $x^p=0$, and since $\F_q(\mu_n)$ contains no zero divisors this only holds for $x=0$.

    To show that the $q$'th power map belongs to $\Gal(\F_q(\mu_n)/\F_q)$, we show it is an automorphism fixing the field precisely equal to $\F_q$.
    The map is an automorphism because it is the $\ell$-depth composition of the Frobenius map, itself an automorphism.
    The $q$'th power map clearly fixes the additive identity of $\F_q$, and it also fixes the multiplicative group $\F_q^\times$ because group order $q-1$ implies $x^{q-1}=1$ for $x\in\F_q^\times$.
    Thus the $q$'th power map fixes $\F_q$, and to see that it \emph{only} fixes the $q$ elements of $\F_q$, note that $X^q-X$ has no more than $q$ roots.
    Furthermore, we can the same about all automorphisms generated by the $q$'th power map. 
    That is, the subgroup of $\Gal(\F_q(\mu_n)/\F_q)$ generated by the $q$'th power map also fixes $\F_q$.

    We now invoke the fundamental theorem of Galois theory to argue that two different subgroups of $\Gal(\F_q(\mu_n)/\F_q)$ must fix different fields.
    By definition, the goup $\Gal(\F_q(\mu_n)/\F_q)$ fixes the field $\F_q$, but so does the subgroup generated by the $q$'th power map.
    Therefore, these groups are in fact the same, that is the $q$'th power map generates $\Gal(\F_q(\mu_n)/\F_q)$.

    Finally we determine the image of the Galois group in $(\Z/n\Z)^\times$.
    Recall that the isomorphism from $\Gal(\F_q(\mu_n)/\F_q)$ to a subgroup of $(\Z/n\Z)^\times$ takes the form $\sigma_k\to(k+\Z)$ where $\sigma_k$ is the $k$'th power map on $\mu_n$ and the identity map on $\F_q$.
    To say the $q$'th power map generates the Galois group is to say $\sigma_q$ generates the Galois group.
    Since $\sigma_q$ maps to $(q+\Z)$, we may conclude that the image is generated by $(q+\Z)$.
    Therefore the Galois group maps to the subgroup $\langle q+\Z\rangle$ of $(\Z/n\Z)^\times$, which has order $\ord_n(q)$.
\end{theorem}

Again invoking the fact that the order of the Galois group is equal to the degree of the Galois extension, we conclude the extension $\F_q(\mu_n)/\F_q$ has degree $\ord_n(q)$.
Now we utilize this fact to examine how $\Phi_n$ splits in the field $\F_q$.

\begin{theorem}[Reducibility of Phi_n over F_q]{Reducibility of $\Phi_n$ over $\F_q$}
    Over the field $\F_q$ the polynomial $\Phi_n$ splits into $\phi(n)/\ord_n(q)$ monic, irreducible factors each of degree $\ord_n(q)$.

    \proof
    By definition, $\Phi_n$ over $\F_q(\mu_n)$ has degree $\phi(n)$ and vanishes on the $\phi(n)$ primitive $n$'th roots of unity.
    Therefore every irreducible factor of $\Phi_n$ over $\F_q$ is the minimal polynomial of one or more primitive $n$'th roots of unity.

    Fix a primitive $n$'th root of unity $\zeta$ with minimal polynomial $f_\zeta$ (an irreducible factor of $\Phi_n$ over $\F_q$).
    The field $\F_q(\zeta)$ is a cyclotomic extension of $\F_q$.
    As implied by the previous theorem, the extension $\F_q(\zeta)/\F_q$ has degree $\ord_n(q)$.
    But this extension is equivalent to $\F_q[X]/f_\zeta(X)$ which has degree $\deg(f_\zeta)$.
    Therefore all irreducible factors of $\Phi_n$ over $\F_q$ have degree $\ord_n(q)$.
    Since $\Phi_n$ has degree $\phi(n)$, there are $\phi(n)/\ord_n(q)$ such factors.
\end{theorem}

We restate the previous theorem for the special case in which $\Phi_n$ is irreducible over $\F_q$.

\begin{corollary}[Irreducibility of Phi_n over F_q]{Irreducibility of $\Phi_n$ over $\F_q$}
    In the field $\F_q$ the polynomial $\Phi_n$ is irreducible if and only if $\ord_n(q)=\phi(n)$.
\end{corollary}


\begin{references}
    \source{Chen06}
    \title{Course notes}
    \authors{Jungkai Alfred Chen}
    \when{2006}
    \where{Department of Mathematics, National Taiwan University}
    \other
    \link[Notes]{www.math.ntu.edu.tw/~jkchen/F06AA/F06AA1L11.pdf}

    \source{Conrad}
    \title{Cyclotomic Extensions}
    \authors{Keith Conrad}
    \when{}
    \where{}
    \other
    \link[Cyclotomic blurb]{kconrad.math.uconn.edu/blurbs/galoistheory/cyclotomic.pdf}

    \source{Mitchell}
    \title{Supplementary field theory notes}
    \authors{Steve Mitchell}
    \when{}
    \where{}
    \other
    \link[math.washington.edu]{sites.math.washington.edu/~mitchell/Algg/field.pdf}

    \source{Mor96}
    \title{Field and Galois Theory}
    \authors{Patrick J. Morandi}
    \when{1996}
    \where{Springer Science & Business Media}
    \other
    \link[Springer Link]{link.springer.com/book/10.1007/978-1-4612-4040-2}
    See the section on Cyclotomic Extensions.

    \source{Nguyen}
    \title{A Note on Cyclotomic Integers}
    \authors{Nicholas Phat Nguyen}
    \when{}
    \where{}
    \other
    \link[arXiv:1706]{arxiv.org/ftp/arxiv/papers/1706/1706.05390.pdf}

    \source{Weintraub}
    \title{Several Proofs of the Irreducibility of the Cyclotomic Polynomials}
    \authors{Steven H. Weintraub}
    \when{}
    \where{}
    \other
    \link[lehigh.edu]{lehigh.edu/~shw2/c-poly/several_proofs.pdf}
\end{references}
