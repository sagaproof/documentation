

\section{Rings}

\subsection{Ideals}

\begin{definition}{Ideals}
    An \emph{ideal} $I$ is a subgroup of the additive group of a ring $R$.
    If $rI\subseteq I$ for every $r\in R$, then $I$ is a \emph{left ideal}.
    If $Ir\subseteq I$ for every $r\in R$, then $I$ is a \emph{right ideal}.
    If both conditions hold, the ideal is \emph{two sided}.
\end{definition}

The fact that ideals must be closed under multiplication with the entire ambient ring reflects how ideals are intended to capture the span of all multiples of some subset.
Commonly one starts with a (commutative) ring subset $\{g_1,\dots,g_n\}\subseteq R$ and considers the ideal consisting of all multiples of the elements, written as
\begin{equation}
    (g_1,\dots,g_n) = \{c_1g_1+\dots+c_ng_n\mid c_1,\dots,c_n\in R\}
\end{equation}
We say such an ideal is \emph{generated} by the set $\{g_1,\dots,g_n\}$.
In ring theory, parentheses surrounding ring elements as such denote the ideal generated by those elements.

The \emph{sum} and \emph{product} of two ideals $I,J$ are defined as the ideals generated by the set-theoretic union and intersection of $I$ and $J$, respectively.
One can prove that these definitions may be expressed as
\begin{align}
    I+J\coloneqq\langle I\cup J\rangle&=\left\{i+j\mid i\in I,\ j\in J\right\}\\
    IJ\coloneqq\langle I\cap J\rangle&=\left\{\sum_{k=1}^n i_kj_k\mid i_k\in I,\ j_k\in J,\ n\in\Z\right\}
\end{align}

Ideals were conceived to generalize certain sets of numbers, and as such they carry an analogue notion of primality, that is what it means for an ideal to be `prime' or for two ideals to be `coprime.'

\begin{definition}{Division and primality in ideals}
    Suppose $I,J\subseteq R$ are two ideals.
    The ideal $I$ \emph{divides} $J$ if $I\subseteq J$.
    The \emph{greatest common divisor} of $I$ and $J$ is $I+J$.
    The ideals $I$ and $J$ are \emph{relatively prime} or \emph{coprime} if $I + J = R$.
    The ideal $I$ is \emph{prime} if $\forall x,y\in R$ such that $xy\in I$, either $x\in I$ or $y\in I$.
\end{definition}

It may be illuminating to consider these notions in the special case of the ring of integers to justify them as natural generalizations of number-theoretic primarily.
Suppose an ideal in $\Z$ is generated by an integer $s$, that is the ideal $\langle s\rangle$ consisting of all multiples of $s$.
Suppose for all $x,y\in\Z$ we have it that $xy\in\langle s\rangle$ implies $x\in\langle s\rangle$ or $y\in\langle s\rangle$.
Then by definition the ideal $\langle s\rangle$ is prime, and by intuition the number $s$ is prime as $s$ cannot divide the product of two numbers unless it divides one of the numbers.
One may go the other direction, assuming primality of $s$ and concluding $\langle s\rangle$ is a prime ideal.
Now suppose $\langle s\rangle$ and $\langle s'\rangle$ are two ideals in $\Z$ such that $\langle s\rangle+\langle s'\rangle=\Z$.
Then by targeting $1\in\Z$ on the right side, we see there exists $t,t'\in\Z$ such that $ts+t's'=1$ and as such $\text{gcd}(s,s')=1$.
On the other hand, if we start with $ts+t's'=1$ we may multiply through by any integer to recover $\langle s\rangle+\langle s'\rangle=\Z$.




\subsection{Ext and Quot}

\begin{definition}{Adjoining elements}
    Given a ring $R$, to \emph{adjoin} elements $x_i$ to $R$ means to consider the ring consisting of elements $R\cup\{x_i\}$ with addition and multiplication operations inherited by those of $R$.
    We refer to $x_i$ as `elements' but the set to which they belong is often left implicit, and when context offers no answer $x_i$ may be interpreted as indeterminates.

    The notation $R[x_1,x_2,\dots]$ denotes the ring had by adjoining elements $x_1,x_2,\dots$ to $R$.
    The notation $R(x_1,x_2,\dots)$ denotes the ring had by adjoining elements $x_1,x_2,\dots$ along with their multiplicative inverses $x_1^{-1},x_2^{-1},\dots$ to $R$.
\end{definition}

\begin{definition}{Quotients}
    
\end{definition}

% If $\alpha$ has a minimal polynomial then $\alpha$ is \emph{algebraic} over $F$; otherwise $\alpha$ is \emph{transcendental} over $F$.
If alpha vanishes on some poly (arith expr) over a commutative ring $R$ then alpha is emph{algebraic} over R; oterhwise, alpha is emph{transcendental} over F.


\begin{definition}{Integral extension}
    An element of a commutative ring $x\in R$ is \emph{integral} over a subring $R'$ if it vanishes on some polynomial over $R'$.

    The ring $R$ is an \emph{integral extension} of subring $R'$ if all elements in $R$ are integral over $R'$.
\end{definition}
(diff notation, R/R' why not,)


\section{Field extensions}

\begin{definition}{Field extension}
    A field $E$ is said to \emph{extend} a field $F$ if $F$ is a subfield of $E$.
    We refer to $E$ in regard to $F$ as a \emph{field extension}, and write $E/F$ though this notation has nothing to do with quotients.
    We may refer to $E$ as an \emph{extension field}, implicity in regard to some subfield $F$.

    We may treat $E$ as a vector space over $F$.
    The \emph{degree} of the field extension, denoted $[E:F]$, is the dimension of the corresponding vector space.
    We may refer to an extension of finite degree as a `finite extension.'
\end{definition}

\begin{definition}{Minimal polynomial}
    Let $E/F$ be a field extension.
    The \emph{minimal polynomial} of an element $\alpha\in E$ (if it exists) is the non-zero, monic polynomial over $F$ of least degree vanishing on $\alpha$.

    If $\alpha\in E$ has a minimal polynomial in $F$ then $\alpha$ is \emph{algebraic} over $F$; otherwise, $\alpha$ is \emph{transcendental} over $F$.

    The \emph{degree} of an algebraic element with respect to an extension $E/F$ is the degree of its minimal polynomial.
\end{definition}

The notion of algebraic elements in a field extension is the analogue of integral elements in a ring extension.
Just as defining integral elements lead to defining integral extensions, having defined algebraic elements we now define algebraic extensions.

\begin{definition}{Algebraic and transcendental extension}
    Consider a field extension $E/F$.
    If all elements of $E$ are algebraic over $F$ then the extension is \emph{algebraic}; otherwise, the extension is \emph{transcendental}.
\end{definition}

If an extension is of finite degree then it is algebraic, as otherwise powers of a transcendental element $\alpha$ would form a basis of infinite dimension.

\begin{definition}{Splitting field}
    A splitting field of a polynomial $p$ over a field $F$, is a extension field $E/F$ in which $p$ splits into linear factors.
\end{definition}

\begin{definition}{Separable extension}
    A polynomial over a field $F$ is \emph{separable} if it has distinct roots in a splitting field.
    
    An extension $E/F$ is \emph{separable} if the minimal polynomial over $F$ of every element in $E$ is separable; otherwise, the extension is \emph{inseparable}.
\end{definition}

\begin{definition}{Normal extension}
    An algebraic extension $E/F$ is \emph{normal} if the minimal polynomial over $F$ of every element in $E$ splits (completely) in $E$.
\end{definition}

\begin{definition}{Galois extension}
    An algebraic extension $E/F$ is a \emph{Galois extension} if it is normal and separable.
    In other words, the minimal polynomial over $F$ of every element in $E$ splits into distinct linear factors in $E$.

    The set $\text{Aut}(E/F)$ of automorphisms associated with $E/F$ is the set of automorphisms on $E$ that fix $F$.
    These automorphisms form a subgroup of the group of automorphisms on $E$.
    The subgroup is referred to as the \emph{Galois group} for $E/F$, and denoted $\Gal(E/F)$.
\end{definition}

The fundamental theorem of Galois theory roughly says in regard to $E/F$ there is a one-to-one correspondence between subfields $F\subseteq K_i\subseteq E$ and subgroups $\iota\subseteq G_i\subseteq\Gal(E/F)$ ($\iota$ the identity function) such that the field fixed by $G_i$ is precisely $K_{\ell-i}$ if there are $\ell$ total fields between $F$ and $E$ (inclusive).
Since the largest group $\Gal(E/F)$ fixes the smallest field $F$, a tower of fields $F=K_0\subseteq K_1\subseteq\dots\subseteq K_\ell=E$ corresponds to an inverted tower of subgroups $\Gal(E/F)=G_\ell\supseteq\dots\supseteq G_1\supseteq G_0=\iota$.
Though we will not prove the fundamental theorem of Galois theory, we also will not hesitate to use it.


\subsection{Number fields}

\begin{definition}{Number field}
    A \emph{number field} is a finite extension of the field of rational numbers $\Q$.
\end{definition}

All number fields can be constructed by adjoining to $\Q$ some number $\alpha\in\C$ algebraic over $\Q$, that is $\Q(\alpha)=\Q[\alpha]$.
The degree of the field extension $\Q(\alpha)/\Q$ is the number of linearly independent powers of $\alpha$, which is the degree of its minimal polynomial $f_\alpha$ (justifying how we defined `degree of $\alpha$' to be $\deg(f_\alpha)$).
An equivalent extension can be constructed by the quotient $\Q[x]/f_\alpha(x)$.
Associating the indeterminate $x$ with $\alpha$ recovers the original formulation.
Yet any other of the $\deg(f_\alpha)$ complex root of $f_\alpha$ could also be substituted for $x$, yielding different but equivalent formulations of the same field.
These $\deg(f_\alpha)$ formulations are called the \emph{embeddings} of the field because they each embed $\Q[x]/f_\alpha(x)$ into $\R$ or $\C$ as a subfield.

\begin{definition}{Ring of integers}
    An \emph{algebraic integer} is some element of $\C$ integral over $\Z$.
    Algebraic integers form a ring extending the integers.

    The \emph{ring of integers} $\mathcal{O}_F$ of a number field $F$, is the intersection of $F$ with the ring of algebraic integers. 
    The notation $\mathcal{O}_F$ reflects how the ring of integers is a .,cular kind of subring called an `order.'
    The ring of integers generalize the integers with respect to $F$ as $\Z\cup\mathcal{O}_F$.
\end{definition}

For example, the ring of integers of $\Q$ is $\mathcal{O}(\Q)=\Z$ because by the `integral root theorem' the only rationals integral over the integers are the integers themselves.

It turns out that rings of integers are instances of `Dedekind domains' which means all non-trivial ideals factor uniquely into a product of prime ideals.
(all prime ideals are maximal)








