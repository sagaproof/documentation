
\begin{theorem}[Polynomial eta-decomposition]{Polynomial $\eta$-decomposition}
    Let $p$ be any polynomial over a commutative ring, and denote the `size' of $p$ by $n=\deg(p)+1$.
    Then $p$ may be decomposed by any non-constant, monic polynomial $\eta$ into $\deg(\eta)$ polynomials $p^{(0)},\dots,p^{(\deg(\eta)-1)}$ all of degree at most $n/\deg(\eta)-1$ such that
    \begin{equation}
        p(x) = \sum_{i=0}^{\deg(\eta)-1} p^{(i)}(\eta(x))x^i
    \end{equation}
    Furthmore, if $\eta$ is a monomial then 
    \begin{equation}
        p^{(i)}(x) = \sum_{j=0}^{n/\deg(\eta)-1} p[i+j\cdot\deg(\eta)]x^j
    \end{equation}
    where the square brackets index into $p$'s coefficients.

    \proof
    First invoke the following recursive algorithm on $p$ to obtain a `transposed' decomposition, from which we will then take the desired decomposition.

    \begin{enumerate}
        \begin{itemize}[square]
            \bold{Decompose:}
            \item
            \bold{Input:} polynomial $t$. Let $n_t=\deg(t)+1$
            \item
            \bold{Output:} polynomials $t^{(0)},\dots,t^{(\ceil{n_t/\eta}-1)}$ with $\deg(t^{(i)})<\deg(\eta)$ for all $i$ such that
            \begin{align}
                t(x) = \sum_{i=0}^{\ceil{n_t/\deg(\eta)}-1} t^{(i)}(x)\eta(x)^i
            \end{align}
            Furthermore, if $\eta$ is a monomial then
            \begin{equation}
                t^{(i)} = \sum_{j=0}^{\deg(\eta)-1} t[j+i\cdot\deg(\eta)]x^j\tag{1}\label{t-form}
            \end{equation}
        \end{itemize}

        \item
        If $\deg(t)<\deg(\eta)$ output $t^{(0)}\coloneqq t$.
        Otherwise, divide $t$ by $\eta$ to get quotient $q$ and remainder $r$ with $\deg(r)<\eta$.
        Note $\deg(q)+\deg(\eta) = \deg(t)$, and that zero-divisors do not spoil division since $\eta$ is monic.

        If $\eta$ is a monomial then the coefficients of $t$ will be distributed among $q$ and $r$ as
        \begin{align}
            q(x) &= \sum_{i=0}^{\deg(t)-\deg(\eta)} t[\deg(\eta)+i]x^i \tag{2}\label{q-form} \\
            r(x) &= \sum_{i=0}^{\deg(t)-1} t[i]x^i \tag{3}\label{r-form}
        \end{align}

        \item
        Let $n_q=\deg(q)+1$.
        Invoke \bold{Decompose} on $q$ to get $q^{(0)},\dots,q^{(\ceil{n_q/\deg(\eta)}-1)}$ with $\deg(q^{(i)})<\deg(\eta)$ for all $i$.

        \item
        For $\alpha\in\R$ we have the identity $\floor{\alpha-1} = \floor{\alpha}-1$, from which we infer
        \begin{align}
            &\deg(q)+\deg(\eta)=\deg(t) \\
            &\implies n_q = n_t-\deg(\eta) \\
            &\implies \ceil{\frac{n_q}{\deg(\eta)}}
                = \ceil{\frac{n_t}{\deg(\eta)}-1}
                = \ceil{\frac{n_t}{\deg(\eta)}}-1
        \end{align}
        Now $t$ may be expressed as
        \begin{align}
            t(x)
            &= r(x) + \eta(x)q(x) \\
            &= r(x) + \eta(x)\sum_{i=0}^{\ceil{n_q/\deg(\eta)}-1} q^{(i)}(x)\eta(x)^i \\
            &= r(x) + \sum_{i=1}^{\ceil{n_q/\deg(\eta)}} q^{(i-1)}(x)\eta(x)^i \\
            &= \sum_{i=0}^{\ceil{n_t/\deg(\eta)}-1} t^{(i)}(x)\eta(x)^i
        \end{align}
        Where in the last expression we assigned $t^{(0)},\dots,t^{(\ceil{n_t/\eta}-1)} \coloneqq r,q^{(0)},\dots,q^{(\ceil{n_q/\eta}-1)}$.
        Thus the decomposition satisfies the promised general form.
        For the case that $\eta$ is a monomial, $t^{(0)}=r$ is correct by way of $\eqref{r-form}$, and $t^{(i)}$ for $i>0$ is correct by the combination of the form of $q^{(i)}$ in $\eqref{t-form}$ and the form of $q$ in $\eqref{q-form}$, yielding
        \begin{align}
            t^{(i)}(x) &= q^{(i-1)} \\
            &= \sum_{j=0}^{\deg(\eta)-1} q[j+(i-1)\deg(\eta)]x^j \\
            &= \sum_{j=0}^{\deg(\eta)-1} t[\deg(\eta)+(j+(i-1)\deg(\eta))]x^j \\
            &= \sum_{j=0}^{\deg(\eta)-1} t[j+i\cdot\deg(\eta)]x^j \\
        \end{align}
    \end{enumerate}

    The current decomposition is a polynomial with respect to $\eta(x)$ of degree less than $\ceil{n_t/\deg(\eta)}$ with coefficients as polynomials with respect to $x$ of degree less than $\eta$. 
    The desired decomposition is a polynomial with respect to $x$ of degree less than $\eta$ with coefficients as polynomials with respect to $\eta(x)$ of degree less than $\ceil{n_t/\deg(\eta)}$.
    Hence the current decomposition may transformed into the desired decomposition by something of a transposition, that is swapping polynomial indices and coefficient indices, that is assigning $p_j^{(i)}\coloneqq t_i^{(j)}$.
\end{theorem}


\begin{theorem}[eta-decomposition by interpolation]{$\eta$-decomposition by interpolation}
    If the preimage of some $\alpha$ under $\eta$ has size at least $\deg(\eta)$ and contains no zero-divisors, then interpolating $p$ over this preimage yields a polynomial with coefficients $p^{(0)}(\alpha),\dots,p^{(\deg(\eta)-1)}(\alpha)$, namely the $\eta$-decomposition polynomials of $p$ on $\alpha$.

    \proof
    Let $Y = \eta^{-1}(\alpha)$, noting $|Y|\geq\eta$.
    Then $p$ restricted to $Y$ yields the polynomial
    \begin{equation}
        p(y) = \sum_{i=0}^{\deg(\eta)-1} p_i(\eta(y))y^i
        = \sum_{i=0}^{\deg(\eta)-1} p_i(\alpha)y^i
    \end{equation}
    This polynomial of degree less than $\deg(\eta)$ may be interpolated over any $\deg(\eta)$ values of $Y$, thus extracting coefficients $p^{(0)}(\alpha),\dots,p^{(\deg(\eta)-1)}(\alpha)$.
\end{theorem}
