
\section{Linear relation proof of knowledge}

Let $\R$ be a commutative ring, $\S$ an $\R$-module, and $A\in\S^m$.
Consider the following sigma protocol between a prover and verifier.
\begin{enumerate}
    \item
    The prover sends $T\in\S^{\ell_1}$ to the verifier.

    \item
    The verifier samples a challenge $C\in\R^{\ell_1\times\ell_2}\leftarrow\C$ and sends it to the prover.

    \item
    The prover replies with $W\in\R^{m\times\ell_2}$, which the verifier then checks satisfies the equation $A^\intercal W=T^\intercal C$.
    The verifier additionally checks $W$ with respect to a function $\Check\colon\R^{m\times\ell_2}\to\{0,1\}$ ensuring $\Check(W)=1$.
\end{enumerate}

We will show the protocol satisfies the following notions of completeness and knowledge soundness.

\bold{Completeness:}

Suppose the prover holds $V\in\R^{m\times\ell_1}$ such that for all possible challenges $C\in\C\subseteq\R^{\ell_1\times\ell_2}$ it holds that $\Check(VC)=1$.
Furthermore, suppose the prover in the first round sends $T\coloneq A^\intercal V$, and in the third round for verifier's challenge $C\in\C$ sends $W\coloneq VC$.
Then the prover always passes.

\bold{Knowledge soundness:}

% $LRE(\iota,\C,\Check,\Clash)$

Suppose for $\iota\in[\ell_1]$ that the $\iota$'th row of the distribution $\C\subseteq\R^{\ell_1\times\ell_2}$ is the transpose of a distribution $\C_\iota\subseteq\R^{\ell_2}$.
Furthermore, suppose for symmetric relation $\Clash\colon\C_\iota\times\C_\iota\to\{0,1\}$ and any $c\in\C_\iota$, the probability $c'\leftarrow\C_\iota$ clashes with $c$ is at most $\rho$, that is
\begin{equation}
    \Pr[\Clash(c,c')=1|c\in\C_\iota, c'\leftarrow\C_\iota] \leq \rho
\end{equation}
Then if the prover passes with probability $\epsilon\geq\rho/\sigma$ for constant $\sigma\in[\rho,1)$ we may construct an extractor that, given oracle access to the prover, runs in expected time $\bigO(1)/\epsilon$ and outputs $W,W'\in\R^{m\times\ell_2}$ and $c,c'\in\C_\iota\subseteq\R^{\ell_2}$ such that the following hold.
\begin{gathered}
    \Check(W) = \Check(W') = 1, \Clash(c,c') = 0 \\
    A^\intercal(W-W') = T_\iota(c-c')^\intercal
\end{gathered}


\subsection{Proving completeness}

To always pass means that regardless the verifier's challenge $C$, both the verifier's checks pass.
With the prover behaving as described we have $\Check(W)=\Check(VC)=1$, and also $A^\intercal W=A^\intercal (VC)=(A^\intercal V)C=T^\intercal C$.


\subsection{Proving knowledge soundness}

Denote the probabilistic prover as a function $\mathcal{P}\colon\C\times\Omega\to\R^{m\times\ell_2}$ where $\Omega$ is the distribution of randomness fed to the prover.
We will invoke the \link[heavy class extractor]{} $HCE(\Sample,\Valid,\Clash)$ parameterized as follows.
\begin{itemize}
    \item
    The heavy class extractor operates with respect to a set on which we define a probability distribution as well as an equivalence relation $\sim$.
    Let the set be $(\C,\Omega)$, and for $(C,\omega),(C',\omega')\in(\C,\Omega)$ let $(C,\omega)\sim(C',\omega')$ if $C_i = C'_i$ for all $\ell_1-1$ rows $i\neq\iota$.

    \item
    Let sampler $\Sample\colon(\C,\Omega)?\to(\C,\Omega)$ accepting optional argument operate as follows.
    \begin{itemize}
        \item
        If no argument is present, sample from $(\C,\Omega)$ and return the result.
        \item
        If an argument $(C,\omega)$ is present, sample from the same equivalence class and return the result.
        To do so, first sample randomness $\omega'\leftarrow\Omega$, and sample an $\iota$'th challenge row $c\leftarrow\C_\iota$.
        Then return $(C',\omega')$, where $C'_i=C_i$ for all $\ell_1-1$ rows $i\neq\iota$, and $C'_\iota=c^\intercal$. 
    \end{itemize}

    \item
    The event of validity $\Valid\colon(\C,\Omega)\to\{0,1\}$ is the event that the prover succeeds on the argument.
    In particular, for $(C,\omega)\in(\C,\Omega)$ we invoke the prover on challenge $C$ and randomness $\omega$ to obtain $W = \mathcal{P}(C,\omega)$.
    If $\Check(W)=1$ and $A^\intercal W = T^\intercal C$ return $1$ and otherwise return $0$.

    While randomness $\omega$ is fed to the prover along with the challenge, only the challenge is visible to the extractor.
    Moreover, the extractor cannot bias the randomness.
    Even requesting to use the same randomness as in the previous invocation is forbidden.

    \item
    We will extend the provided clashing relation $\Clash$ to obtain another clashing relation $\ClashExt$ for the heavy class extractor.
    Specifically, let $\ClashExt\colon(\C,\Omega)\times(\C,\Omega)\to\{0,1\}$ be an extension of $\Clash\colon\C_\iota\times\C_\iota\to\{0,1\}$ defined as
    \begin{align}
        \ClashExt((C,\omega),(C',\omega')) \coloneq \Clash(C_\iota,C'_\iota)
    \end{align}
    where $C_\iota$ and $C'_\iota$ are the $\iota$'th rows of $C$ and $C'$.
\end{itemize}

We now prove the assumptions necessary for the heavy class extractor by using our assumptions for this protocol.
The first assumption we must prove is with respect to validity, that is a lower bound on the probability $\Valid((C,\omega))=1$ for $(C,\omega)\leftarrow(\C,\Omega)$.
The function $\Valid$ invokes the prover on challenge $C$ and randomness $\omega$, checks the prover's response, and outputs $1$ if the prover passes and $0$ otherwise.
The probability $\Valid$ outputs $1$ is precisely the probability the prover passes, and since we assume the prover passes with probability $\epsilon\geq\rho/\sigma$ we conclude
\begin{equation}
    \Pr[\Valid((C,\omega))=1|(C,\omega)\leftarrow(\C,\Omega)]
        = \epsilon \geq \rho/\sigma
\end{equation}

The second assumption we must prove is with respect to clashing, that is an upper bound on the probability $\ClashExt((C,\omega),(C',\omega'))=1$ for any $(C,\omega)\in(\C,\Omega)$ and $(C',\omega')\leftarrow(\C,\Omega)$.
By the definition of $\ClashExt$ this occurs only when $\Clash(C_\iota,C'_\iota)=1$ so we conclude
\begin{gathered}
    \Pr[
        \ClashExt((C,\omega),(C',\omega'))
        | (C,\omega)\in(\C,\Omega), (C',\omega')\leftarrow(\C,\Omega)
    ]
    = \Pr[
        \Clash(C_\iota,C'_\iota)
        | C\in\C, C'\leftarrow\C, \omega\in\Omega, \omega'\leftarrow\Omega
    ]
    = \Pr[
        \Clash(C_\iota,C'_\iota)
        | C\in\C, C'\leftarrow\C
    ]
    = \Pr[
        \Clash(C_\iota,C'_\iota)
        | C_\iota\in\C_\iota, C'_\iota\leftarrow\C_\iota
    ]
    \leq \rho
\end{gathered}

We now construct our promised output using the promised output of the heavy class extractor.
The heavy class extractor returns $(C,\omega)$ and $(C',\omega')$ in the same equivalence class of $(\C,\Omega)/\sim$ such that both are valid with respect to $\Valid$ and they do not clash with respect to $\ClashExt$, that is
\begin{gathered}
    \Valid((C,\omega)) = \Valid((C',\omega')) = 1 \\
    \ClashExt((C,\omega),(C',\omega')) = 0
\end{gathered}
By the definition of $\Valid$ we conclude that $W=\mathcal{P}(C,\omega)$ and $W'=\mathcal{P}(C',\omega')$ satisfy
\begin{align}
    \Check(W) = \Check(W') = 1 \\
    A^\intercal W = T^\intercal C, A^\intercal W' = T^\intercal C'
\end{align}
By the definition of $\ClashExt$ we conclude
\begin{equation}
    \Clash(C_\iota,C'_\iota) = \ClashExt((C,\omega),(C',\omega')) = 0
\end{equation}
Since $C$ and $C'$ are in the same equivalence class, for the $\ell_1-1$ rows $i\neq\iota$ we have $C_i=C'_i$, implying
\begin{equation}
    A^\intercal(W-W') = T^\intercal(C-C')
    = \sum_{i\in[\ell_1]} T_i(C_i-C'_i)^\intercal
    = T_\iota(C_\iota-C'_\iota)^\intercal
\end{equation}
The output of our extractor is $W,W'\in\R^{m\times\ell_2}$ and $C_\iota,C'_\iota\in\C_\iota$.

Lastly, since the heavy class extractor operates in expected time $\bigO(1)/\epsilon$, so does our extractor.










in both examples below we'll have H sample each element in R^l2 iid, and in Clash we'll have the conjunction of the same predicate for all elements.
we leave the rows alone to emphasize our method of extraction doesn't depend on independent entries in each row, not just for sampling the rows but for defining their clashing



\section{Discrete Log Example}

Let the ring $\R$ be a finite field $\F_q$ of prime order $q$, and let the $\R$-module $\S$ be a group $\G$ of order $q$.
Let $A\in\G^m$ be a vector of group elements.
A prover and verifier may engage in the following sigma protocol.
\begin{enumerate}
    \item
    The prover sends $T\in\G^\ell$ to the verifier.

    \item
    The verifier samples a challenge $C\in\F_q^\ell$ uniformly and sends it to the prover.

    \item
    The prover replies with $W\in\F_q^m$, which the verifier then checks satisfies the equation $A^\intercal W=T^\intercal C$.
\end{enumerate}

The protocol satisfies the following notions of completeness and knowledge soundness.

\bold{Completeness:}

Suppose the prover holds $V\in\F_q^{m\times\ell}$.
Furthermore, suppose the prover in the first round sends $T\coloneq A^\intercal V$, and in the third round for verifier's challenge $C\in\F_q^\ell$ sends $W\coloneq VC$.
Then the prover always passes.

\bold{Knowledge soundness:}

If the prover passes with probability $\epsilon\geq1/(\sigma q)$ for constant $\sigma\in[1/q,1)$ we may construct an extractor that, given oracle access to the prover, runs in expected time $\bigO(1)/\epsilon$ and outputs $V\in\F_q^{m\times\ell}$ such that $A^\intercal V = T^\intercal$.



Proof
Completeness:
Completeness follows from the completeness of the \link[linear relation proof of knowledge]{}.


Knowledge soundness:

For each $\iota\in[\ell]$ we will apply the knowledge soundness result from \link[linear relation proof of knowledge]{} to extract the $\iota$'th column of promised output $V\in\F_q^{m\times\ell}$.
We apply the knowledge soundness result using the following parameters.
\begin{itemize}
    \item
    $\ell_1\coloneq\ell$ and $\ell_2\coloneq1$

    \item
    $\C_\iota\coloneq\F_q$ is the uniform distribution on $\F_q$.

    \item
    $\Check(W\in\F_q^m)\coloneq1$ is trivially accepting.

    \item
    $\Clash(x,y)\coloneq x=y?1:0$ is the trivial equality relation.
    Then for any $x\in\F_q$ the probability $y\leftarrow\F_q$ clashes with $x$ is $1/q$.
\end{itemize}
Thus in expected time $\bigO(1)/\epsilon$ the extractor outputs $W,W'\in\F_q^m$ and $c\neq c'\in\F_q$ such that $A^\intercal(W-W') = T_\iota(c-c')$.
Setting the $\iota$'th column of $V$ to $(W-W')/(c-c')$ we have $A^\intercal V = T^\intercal$.



\section{Short Integer Solution Example}

Let $\R=\Z_q$ for integer $q>0$ and let $\S=\Z_q^n$ and $A\in\Z_q^{m\times n}$.
we invoke the linear relation proof of knowledge for all iota in ell_1
C_iota is uniform on {0,1}
Check(W) is |W| <=? beta
Clash(c,c') is c!=?c'
Output we get W,W' and c,c' st. A(W-W') = T_iota(c-c')
If T_iota is zero everywhere then A(W-W') = T_iota = 0
otherwise pick index where c!=c' to get d, then A(W-W')/d = T_iota









