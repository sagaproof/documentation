\begin{define}
    \newcommand\Sample{\mathtt{Sample}}
    \newcommand\Clash{\mathtt{Clash}}
    \newcommand\Valid{\mathtt{Valid}}
\end{define}

This page presents the construction and analysis for a knowledge extractor that is to be used as a subroutine by some other extractor.
It extracts two solutions related in a particular way, which is of use to the calling-extractor.
The construction is inspired by a similar method in \cite{Dam10} where it is used in the discrete log setting.

\begin{references}
    \source{Dam10}
    \title{On Σ-protocols}
    \authors{Ivan Damgard}
    \when{2010}
    \where{}
    \other
    \link{cs.au.dk/~ivan/Sigma.pdf}
\end{references}

\begin{theorem}{Heavy Class Extractor}
    The heavy class extractor $HCE(\Sample,\Valid,\Clash)$ is a sub-extractor to be used by other extractors.
    It serves the purpose of extracting two related solutions, which the calling-extractor can then use as it pleases.
    The extractor is parameterized as follows.
    \begin{enumerate}
        \item
        Let $S$ be a set on which we have defined a probability distribution as well an equivalence relation $\sim$.
        We henceforth refer to equivalence classes as simply `classes.'
        The extractor does not need access to $S$ or $\sim$ but instead interacts with the following helper functions.
        As such, $S$ and $\sim$ are implicit parameters, hence their absence in the parameter list.

        \item
        Instead of sampling directly from the distribution on $S$, we sample via the helper function $\Sample\colon S?\to S$ where the optional argument serves as a representative for a class.
        If an argument $x\in S$ is present, sampling is restricted to the class $C\in S/\sim$ for which $x\in C$.
        Otherwise, sampling is performed on the entire set $S$.

        We provide the extractor access to $\Sample$ rather than $\sim$ (and thus $S/\sim$) because sampling restricted to a particular class may not be straightforward.
        Also note that while $\Sample$ is probabilistic, it is a function and not a distribution so in sampling $x$ we write $x=\Sample()$ and not $x\leftarrow\Sample$.

        \item
        The helper function $\Valid\colon S\to\{0,1\}$ defines an event of `validity,' that is a subset $\Valid(S)\coloneq\{x\in S|\Valid(x)=1\}\subseteq S$ of `valid' outcomes.
        The goal of the extractor is to find two valid outcomes in the same class, provided those two outcomes don't `clash' as described next.

        \item
        For each class $C\in S/\sim$ a helper function $\Clash\colon C\times C\to\{0,1\}$ may be invoked.
        The function defines a symmetric relation (likely but not necessarily an equivalence relation) indicating if two challenges `clash' or if they may serve as solutions.
        Specifically, two valid outcomes in the same class that don't clash may serve as solutions.
    \end{enumerate}

    Suppose the following inequalities hold with regard to parameters $\epsilon\in(0,1]$, $\rho\in(0,1)$ and $\sigma\in[\rho,1)$.
    \begin{gathered}
        \Pr[\Valid(x)=1|x\leftarrow S] = \epsilon \geq \rho/\sigma \\
        \forall C\in S/\sim,\forall x\in C,
            \Pr[\Clash(x,y)|y\leftarrow C] \leq \rho
    \end{gathered}
    Then we may construct an extractor that outputs $x,y\in C$ for some $C\in S/\sim$ such that
    \begin{gathered}
        \Valid(x) = \Valid(y) = 1 \\
        \Clash(x,y) = 0
    \end{gathered}
    Moreover, the extractor runs in expected time
    \begin{equation}
        \frac{\mathcal{O}(1)}{\epsilon}
    \end{equation}

    \proof
    We prove the theorem by constructing such an extractor in the sections that follow.

\end{theorem}

We first present a lemma that plays a core role in analysis.
After presenting the construction of the extractor, we analyze its correctness and expected running time.
The extractor consists of repeating a subroutine as many times as necessary to succeed.
We will show the subroutine takes expected time $\mathcal{O}(1/\epsilon)$, and that is succeeds with probability bounded above by some constant. 
???
The subroutine must then only be executed an expected constant number of times to succeed, thus preserving the expected time $\mathcal{O}(1/\epsilon)$.


\section{Heavy Class Lemma}

Our analysis will center around the following lemma which generalizes that from \cite{Dam10} beyond the special case of $t=1/2$, and beyond the uniform distribution.

\begin{lemma}{Heavy Class Lemma}
    Consider a set $S$ on which we define a probability distribution as well an equivalence relation $\sim$.
    Suppose a particular event $E\subseteq S$ occurs with probability $\epsilon$.

    For $t\in[0,1)$, define the \bold{heavy} classes of $S/\sim$ to be those equivalence classes in which $E$ occurs with (conditional) probability more than $\epsilon t$.
    That is, class $C\in S/\sim$ is heavy if $\Pr[x\in E|x\leftarrow C]>\epsilon t$.
    Then $E$ occurs in a heavy class with probability at least $(1-t)/(1-\epsilon t)$, that is
    \begin{equation}
        \Pr[\text{Heavy}(C)=\text{true}|x\leftarrow E, x\in C\in S/\sim]
        \geq \frac{1-t}{1-\epsilon t}
    \end{equation}
    In other words, with probability at least $(\epsilon-\epsilon t)/(1-\epsilon t)$ a sample $x\leftarrow S$ is both in $E$ and in a heavy class.

    \proof
    Event $E$ has probability mass $\epsilon$ distributed across $S$.
    Referring to this `event mass' we will examine how it is distributed with respect to equivalence classes.
    We aim to show the event mass across heavy classes is at least $(\epsilon-\epsilon t)/(1-\epsilon t)$.
    Let $\alpha$ be the event mass across heavy classes, and $\beta$ the event mass across non-heavy classes, noting $\alpha+\beta=\epsilon$.
    That is, let
    \begin{align}
        \alpha &\coloneq \Pr[
            x\in E \land\text{Heavy}(C)=\text{true}
            |x\leftarrow S, x\in C\in S/\sim
        ] \\
        \beta &\coloneq \Pr[
            x\in E \land\text{Heavy}(C)=\text{false}
            |x\leftarrow S, x\in C\in S/\sim
        ]
    \end{align}
    By definition of heavy classes, the probability $E$ occurs when restricting to any non-heavy class is at most $\epsilon t$.
    Therefore the probability $E$ occurs when restricting to the union of all non-heavy classes is also at most $\epsilon t$.
    We can upper bound the event mass across non-heavy classes as $\beta\leq\epsilon t\mu$ where $\mu$ is the probability mass (not necessarily event mass) across non-heavy classes.
    Now the event mass across heavy classes is at most all remaining probabiltiy mass, that is $1-\mu$.
    Thus the event mass across all classes is at most $\epsilon t\mu+(1-\mu)$.
    With total event mass $\epsilon$ we must have $\epsilon\leq\epsilon t\mu+(1-\mu)$ also written $\mu\leq(1-\epsilon)/(1-\epsilon t)$.
    The upper bound $\beta\leq\epsilon t\mu\leq\epsilon t(1-\epsilon)/(1-\epsilon t)$ finally implies the lower bound
    \begin{equation}
        \alpha = \epsilon - \beta
        \geq \epsilon - \epsilon t\frac{1-\epsilon}{1-\epsilon t}
        = \frac{\epsilon - \epsilon t}{1 - \epsilon t}
    \end{equation}
\end{lemma}

We will invoke the heavy class lemma substituting identically named $S$ and $\sim$, and substituting $\Valid(S)$ for the event $E$.
The heavy classes of $S/\sim$ are then well defined with respect to some value $t\in[0,1)$ to be chosen later.


\section{Extractor Construction}

The extractor will operate by repeating a subroutine as many times as needed to succeed.
The subroutine will operate roughly as follows.
Sample outcomes until finding a valid outcome.
Then sample other outcomes in the same class for a certain amount of time, such that if the class is heavy we have noticeable probability of finding a valid outcome not clashing with the first, yet if the class is not heavy we don't squander too much time searching.
We repeat this two step process until succeeding.
As for how long to search a class before aborting, the time should be proportional to $1/\epsilon$.
The smaller the probability we succeed, the longer we are allowed to search.
Thus we would like to sample proportional to $\delta/\epsilon$ times for some constant integer $\delta$ to be chosen later.
But the value $\epsilon$ is unknown to the extractor, so we cannot compute $\delta/\epsilon$ directly.
Instead, we will approximate $1/\epsilon$ by sampling random outcomes, and from that approximate $\delta/\epsilon$.

More formally, our subroutine operates as follows.
\begin{enumerate}
    \item
    Sample $x=\Sample()$, then if $\Valid(x)=1$ continue to the next step, otherwise repeat this step.

    \item
    Sample $y=\Sample()$, and throw a coin that lands heads with probability $1/\delta$ (e.g. a $\delta$-sided die).
    If $\Valid(y)=1$ and the coin lands heads, abort the subroutine.

    \item
    Sample $z=\Sample(x)$, then if $\Clash(x,z)=0$ exit with solution $\{x,z\}$, otherwise return to the previous step.
\end{enumerate}


\section{Running Time}

The subroutine operates in expected time $\bigO(1/\epsilon)$.
The first step takes expected time $1/\epsilon$ since outcome $x\in S$ is valid with probability $\epsilon$.
The second and third steps each take constant time, and are executed an expected $\delta/\epsilon$ number of times, since the joint probability outcome $y\in S$ is valid and the coin lands heads is $\epsilon\times1/\delta$.
Therefore the subroutine takes expected time $(1+\delta)/\epsilon$ = $\bigO(1/\epsilon)$.

In the next subsection we will prove the subroutine succeeds with no less than constant probability.
Therefore, the subroutine need only be run an expected constant number of times (the reciprocal of the success probability) to succeed.
Thus the extractor's expected running time is $\bigO(1/\epsilon)$.


\section{Subroutine Success}

We now prove the subroutine succeeds with probability bounded below by some constant.
To do so we lower bound the probability of each of the following three events.
\begin{itemize}
    \item
    In the first step, upon finding a valid outcome, that outcome is in a heavy class.

    \item
    The third step is executed at least $\delta\gamma/\epsilon$ times for some constant $\gamma\in(0,1]$ such that $\delta\gamma\geq1$.
    That is, the second step doesn't abort for at least $\delta\gamma/\epsilon$ executions.
    As $\delta\gamma/\epsilon$ is probability fractional we are implicitly speaking of $\floor{\delta\gamma/\epsilon}$ executions, which is non-zero since $\delta\gamma\geq1\geq\epsilon$.

    \item
    In the case the first valid outcome is in a heavy class, the third step succeeds in finding an valid outcome not clashing with the first after no more than $\delta\gamma/\epsilon\geq1$ executions.
\end{itemize}
If all three events occur, the subroutine succeeds.
Therefore the product of these probabilities lower bounds the probability the subroutine succeeds.

\begin{itemize}
    \item
    First we lower bound the probability the first valid outcome in is a heavy class.
    Recall we \link[previously]{#Heavy-Class-Lemma} defined the heavy classes of $S/\sim$ to be those in which an outcome is valid with (conditional) probability more than $\epsilon t$.
    The heavy class lemma asserts that on sampling a valid outcome, the outcome is in a heavy class with probability at least $(1-t)/(1-\epsilon t)$.
    Since $t$ is constrained to the interval $[0,1)$ this probability is non-zero.

    \item
    Second we must lower bound the probability the third step executes at least $\delta\gamma/\epsilon$ times.
    To do so we lower bound the probability the second step executes at least $\delta\gamma/\epsilon$ times before aborting.
    The coin lands heads with probability $1/\delta$, while the outcome is valid with probability $\epsilon$.
    Both events occur with probability $\epsilon/\delta$, so the probability neither event occurs in the first $\floor{\delta\gamma/\epsilon}$ executions is $(1-\epsilon/\delta)^\floor{\delta\gamma/\epsilon}$ which is larger than the simpler form $(1-\epsilon/\delta)^{\delta\gamma/\epsilon}$ so we consider the latter.
    Comparing with the function $(1-X)^{\gamma/X}$ for $X$ in $[0,1]$, we observe this expression is a decreasing function of $\epsilon$ with minimum value zero at $\epsilon = \delta$.
    Therefore the expression is positive for all $\epsilon < \delta$.
    Since $\epsilon$ is at most $1$, we must constrain $\delta > 1$.
    The minimum value of this expression (at $\epsilon = 1$) is then $(1-1/\delta)^{\delta\gamma}$.

    \item
    Third we will lower bound the probability the third step succeeds within $\delta\gamma/\epsilon$ executions assuming the valid outcome $x\in S$ found in the first step is in a heavy class.
    Suppose $x$ is in heavy class $C\in S/\sim$.
    The third step samples outcomes from $C$, and by assumption the probability a sampled outcome $y\in C$ clashes with $x\in C$ is at most $\rho$, that is
    \begin{equation}
        \Pr[\Clash(x,y)|y\leftarrow C] \leq \rho
    \end{equation}
    Also recall that since $C$ is a heavy class, the probability of a valid outcome when sampling from $C$ is more than $\epsilon t$, that is
    \begin{equation}
        \Pr[\Valid(y)=1|y\leftarrow C] > \epsilon t
    \end{equation}
    Therefore when sampling from $C$, the probability an outcome $y$ is valid but not clashing with $x$ is more than $\epsilon t - \rho$, that is
    \begin{equation}
        \Pr[\Valid(y)=1\land\Clash(x,y)=0]|y\leftarrow C]
        \geq \Pr[\Valid(y)=1|y\leftarrow C] - \Pr[\Clash(x,y)|y\leftarrow C]
        > \epsilon t - \rho
    \end{equation}
    To lower bound the probability the third step succeeds within $\delta\gamma/\epsilon$ executions, we instead upper bound the probability of the complement, that is the probability of failing at least $\delta\gamma/\epsilon$ times which is no more than
    \begin{equation}
        \left(1 - (\epsilon t - \rho)\right)^\floor{\delta\gamma/\epsilon}
    \end{equation}
    This expression lies below 1 for $\epsilon t > \rho$, thus our constraint on $t$ becomes $t\in(\rho/\epsilon,1)$.
    The third step then finds a valid, non-clashing outcome with probability at least
    \begin{equation}
        1 - \left(1 - (\epsilon t - \rho)\right)^\floor{\delta\gamma/\epsilon}
    \end{equation}
\end{itemize}

We take the product of the three lower bounds established above to lower bound the probability they all occur.
Thus the subroutine succeeds with probability at least
\begin{equation}
    \frac{1-t}{1-\epsilon t}\cdot
    \left(1-\frac{1}{\delta}\right)^{\delta\gamma}\cdot
    \left(1 - \left(1 - (\epsilon t - \rho)\right)^\floor{\delta\gamma/\epsilon}\right)
\end{equation}
Recall our constraints to ensure the probability is non-zero are $\delta > 1$, $\gamma\in(0,1]$, $\delta\gamma\geq1$, and $t\in(\rho/\epsilon,1)$.

While the second component is clearly a constant, we must lower bound the first and third probabilities by constants since they depend on non-constant values $\epsilon$ and $\rho$.
As $t$ is constrained to $(\rho/\epsilon,1)$, suppose we choose $t\coloneq\rho/\epsilon+\alpha$ for a constant $\alpha\in(0,1-\rho/\epsilon)$.
Then the first component $(1-t)/(1-\epsilon t)$ becomes an increasing function of $\epsilon$.
Plugging in $\rho/\sigma$ for $\epsilon$ we conclude that for all $\epsilon\geq\rho/\sigma$ the component reduces to constant value at least
\begin{equation}
    \frac{1 - (\sigma + \alpha)}{1 - \rho(\sigma + \alpha)/\sigma}
\end{equation}
To ensure this value is non-zero we must further constrain $\alpha$ to the interval $(0,1-\sigma)$ noting this is a subinterval of $(0,1-\rho/\epsilon)$  by the assumption $\epsilon\geq\rho/\sigma$.
This value is an increasing function of $\rho$ with minimum value $1-(\sigma+\alpha)$ at $\rho=0$.
Smaller values of $\alpha$ yield larger probability values.

Now with our choice of $t=\rho/\epsilon+\alpha$ the third component reduces as
\begin{equation}
    1 - \left(1 - (\epsilon t - \rho)\right)^\floor{\delta\gamma/\epsilon}
    = 1 - \left(1 - \epsilon\alpha\right)^\floor{\delta\gamma/\epsilon}
\end{equation}
Lower bounding the exponent $\floor{\delta\gamma/\epsilon}$ as 
\begin{equation}
    \floor{\delta\gamma/\epsilon}
    = \floor{(1/\epsilon\alpha)\times(\delta\gamma\alpha)}
    \geq \floor{1/\epsilon\alpha}\times\floor{\delta\gamma\alpha}
\end{equation}
and taking advantage of the identity
\begin{equation}
    (1-X)^\floor{1/X} < 1/2 \text{ for } X\in[0,1]
\end{equation}
the third component is lower bounded by $1-1/2^\floor{\delta\gamma\alpha}$.
In contrast to the first component, \emph{larger} values of $\alpha$ yield larger probability values.
One may choose an optimal value for $\alpha$ depending on the values of $\sigma$, $\delta$ and $\gamma$.

Finally the total lower bound becomes the constant
\begin{equation}
    \left(1 - (\sigma + \alpha)\right)\cdot
    \left(1-\frac{1}{\delta}\right)^{\delta\gamma}\cdot
    \left(1 - \frac{1}{2^\floor{\delta\gamma\alpha}}\right)
\end{equation}
